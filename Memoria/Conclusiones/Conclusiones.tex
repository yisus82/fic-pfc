\chapter{Conclusiones y trabajo futuro}
A continuaci�n se muestran las principales l�neas de mejora que podr�an llevarse a cabo en un futuro.

\begin{itemize}
	\item \textbf{Creaci�n de una aplicaci�n de administraci�n:} Se podr�a crear una sencilla aplicaci�n de administraci�n de la aplicaci�n pueda borrar usuarios o administrar m�s f�cilmente la aplicaci�n. En estos momentos, el usuario ``admin'' puede borrar mapas, WMSs o puntos de informaci�n de otros usuarios, pero tiene que hacerlo uno a uno ya que no existe una opci�n de borrado en masa, que podr�a ser deseable en un futuro.
	\item \textbf{Internacionalizaci�n de la aplicaci�n:} La aplicaci�n est� pensada para que sea f�cil su internacionalizaci�n, ya que para mostrar los mensajes se hace uso del soporte para la internacionalizaci�n de Struts, que utiliza un fichero .properties con el contenido de los mensajes en un determinado idioma. Si se quisiera a�adir, por ejemplo, el idioma ingl�s lo �nico que deber�a hacerse es traducir el contenido del fichero \emph{Messages.properties} y grabarlo como \emph{Messages\_en.properties}, as� como implementar una manera de escoger el idioma de la aplicaci�n.
	\item \textbf{Mejora del rendimiento:} Se podr�a intentar mejorar el rendimiento a la hora de mostrar los mapas, haciendo que, en lugar de dividir la zona a mostrar en nueve subzonas y hacer nueve peticiones en paralelo, podr�a dividirse en m�s zonas. Tambi�n se podr�a hacer que el renderizado del mapa con las im�genes resultantes sea m�s efectivo.
	\item \textbf{Internacionalizaci�n de la base de datos:} Se podr�a hacer que el contenido de la base de datos est� disponible en varios idiomas, para que todo el contenido visual de la aplicaci�n web sea internacionalizable en los
idiomas soportados.
	\item \textbf {A�adir nuevas funcionalidades:} Por �ltimo, se podr�an a�adir nuevas funcionalidades, tales como el uso de RSS para obtener los �ltimos mapas, POIs o WMSs incluidos, o mejorar la interfaz de la aplicaci�n.
\end{itemize}
